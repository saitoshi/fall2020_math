% DOCUMENT FORMATING
\documentclass[12pt]{article}
\usepackage[margin=1in]{geometry}

% PACKAGES
\usepackage{amsmath} % For extended formatting
\usepackage{amssymb} % For math symbols
\usepackage{amsthm} % For proof environment
\usepackage{array} % For tables
\usepackage{enumerate} % For lists
\usepackage{extramarks} % For headers and footers
\usepackage{blindtext}
\usepackage{fancyhdr} % For custom headers
\usepackage{graphicx} % For inserting images
\usepackage{multicol} % For multiple columns
\usepackage{verbatim} % For displaying code
\usepackage{tkz-euclide}
\usepackage{pgfplots}
\newtheorem{theorem}{Theorem}[section]
\newtheorem*{theorem*}{Theorem}
\newtheorem{corollary}{Corollary}[theorem]
\newtheorem{lemma}[theorem]{Lemma}

% SET UP HEADER AND FOOTER
\pagestyle{fancy}
\lhead{\MyCourse} % Top left header
\chead{\MyTopicTitle} % Top center header
\rhead{\MyAssignment} % Top right header
\lfoot{\MyCampus} % Bottom left footer
\cfoot{} % Bottom center footer
\rfoot{\MySemester} % Bottom right footer
\renewcommand\headrulewidth{0.4pt} % Size of the header rule
\renewcommand\footrulewidth{0.4pt} % Size of the footer rule
% ----------
% TITLES AND NAMES 
% ----------

\newcommand{\MyCourse}{Math 412}
\newcommand{\MyTopicTitle}{Practice Problems}
\newcommand{\MyAssignment}{Abstract Algebra}
\newcommand{\MySemester}{Fall 2020}
\newcommand{\MyCampus}{University of Hawaii at Manoa}
\begin{document}
\subsection*{Arithmetic in Z Revisited}
\begin{enumerate}
    \item Find the quotient q and remainder r when a is divided by b without using technology 
    \begin{itemize}
        \item[(a)] $ a = 17; b = 4$ \\
        17 = 4(4) + 1 \\
        then it follows that q = 4 and r = 1
        \item[b)] $a = 0 ; b = 19 $ \\
        0 = 19(0) + 0 \\
        q = 0 and r = 0 
        \item[(c)] $a = -17; b = 4$ \\
        -17 = (-5)(4) + 3 \\
        Therefore, q = -5 and r = 3. 
    \end{itemize}
    \item Find the quotient q and remainder r when a is divided by b without using technology 
    \begin{itemize}
        \item[(a)] $a = -51 ; b = 6$ \\
        -51 = 6(-9) + 3 \\
        Therefore, r = 3 and q = -6
        \item[(b)] a = 302 and b = 19 \\
        a = bq + r \\
        302 = (19)(1
        \item[(c)] a = 2000 and b = 17 
    \end{itemize}
    \item (1.1 6) Let a be any integer and b and c be any positive integers. Suppose that when q is divided by c the quotient is k. Prove that when a is divided by bc then the quotient is also k. \\
    Given that when q is divided by c then quotient is k. Then by the division algorithm then there exists a remainder r such that 
    \begin{equation*}
        q = c(k) + r'
    \end{equation*}
    such that $0 \leq r' < c$ \\
    Since $a = b q + r$ from number 5 such that $0 \leq r < b$. 
    \begin{equation*}
        a = b (c(k) +r') + r 
    \end{equation*}
    $\to$
    \begin{equation*}
        a = b c k + b r' + r
    \end{equation*}
    $\to $
    \begin{equation*}
        a = (bc) k + (br' + r)
    \end{equation*}
    \item (1.1 10) Let n be a positive integer. Prove that a and c leave the same remainder when divided by n if and only if $a - c = n k$ for some integer k. \\
    $\longrightarrow$
    Suppose that a and c have the same remainder when divided by n. Then there exists a non-negative integers $q_1, q_2$ such that 
    \begin{equation*}
        a = nq_1 + r
    \end{equation*}
    and 
    \begin{equation*}
        c = n q_2 + r 
    \end{equation*}
    Then from subtracting a - c it yields that 
    \begin{equation*}
        a - c = nq_1 + r - nq_2 - r = n(q_1 - q_2)
    \end{equation*}
    Letting k = $q_1 - q_2$ gives as the result desired. \\
    $\longleftarrow$ 
    \item (1.2 EX 1 )Find the greatest common divisors of the given number. 
    \begin{itemize}
        \item[(a)] (56,72) \\
        72 = 56(1) + 16 \\
        56 = 16(3) + 8 \\
        16 = 8(2) \\
        (56,72) = 8 
        \item[(b)] (24,138) \\
        138 = 24(5) + 18 \\
        24 = 18(1) + 6 \\
        18 = 6(3) \\
        Therefore, (24,138) = 6 
        \item[(c)] (112, 57) \\
        112 = 57(1) + 55 \\
        57 = 55(1) + 2 \\
        55 = 2(27) + 1 \\
        2 = 1(2) \\
        Therefore, the gdc(112,57) = 1
    \end{itemize}
    \item (1.2 EX 4) 
    \begin{itemize}
        \item[(a)] If $a|b$ and $a|c$, prove that $a|(b+c)$. \\
        Suppose $a|b$ and $a|c$. Then there exists an integer m and n such that $a|b = m$ and $a|c = n$. Then by definition of divisibility from $a|b = m$ then b = am and c = an. \\
        Then b + c = am + an = a(m+n) which can be rewritten as $a|(b+c)$ as desired. 
        \item[(b)] If $a|b$ and $a|c$, prove that $a|(br + ct)$ for any $r,t \in \mathbb{Z}$. \\
        Suppose $a|b$ and $a|c$. Then there exists an integer m and n such that $a|b = m$ and $a|c = n$. \\
        Consider taking br + ct. Substitute b and c with what was found in the previous step then $br + ct = (am)r + (an)t = a(mr+ nt)$. Since $mr + nt \in \mathbb{Z}$ then by using the definition of divisibility it can be rewritten as $a|(br + ct)$ as desired. 
    \end{itemize}
    \item (1.2 EX 6) If $a|b$ and $c|d$, prove that $ac|bd$. \\
    Suppose that $a|b$ and $c|d$. Then there exists an integer m and n such that $a|b = m$ and $c|d = n$. \\
    From definition of divisibility it can be concluded that b = am and d = cn. Suppose we take the product of bc. 
    Then $b d = a m c n $ and since taking a product is commutative then $b d = a m c n = (ac)(mn)$ which follows to $ac |bd$. \\
    \item (1.2 EX 8) Prove that (n, n + 1) = 1 for every integer n. \\
    We will prove this by using proof by induction. \\
    \textbf{Base Case n = 1}  \\
    Let n = 1. Then (1 , 1 + 1) = (1 , 2). The using the division algorithm. \\
    2 = 1(2) \\
    We see that the GCD of (1,2) = 1 so base case is met. \\
    \textbf{Inductive Step} \\
    Let P(n) be the preposition of (n, n+1) = 1. Let us assume that the statement P(k) holds true such that (k, k+1) = 1. Then prove that P(k+1) holds to be true.  \\
    $(k+1, k+2) = 1$ \\
    $(k+2) = (k+1)(1) + 1$ \\
    $(k+1) = (1)(k+1)$ \\
    Therefore, the GCD of $(k+1, k+2)$ from the division algorithm is 1. Thus inductive step is met. \\
    Therefore, from proof by induction we have proved that $\forall n \in \mathbb{Z}$ that this is true. 
\end{enumerate}
\end{document}
